\documentclass{article}

\usepackage[a4paper, total={6.5in, 11in}]{geometry}

\PassOptionsToPackage{linesnumbered, boxed, noline}{algorithm2e}
\usepackage{latex/common}

\begin{document}

\section{Broadcast/Convergecast}

Every process should send GO/BACK message with the ID of the root, and the root will send GO with itself. Now, all of the variables($parent, children, expected\_msg, val\_set$) will store the data in a 2-D array where the first index indicates the different roots. \\

\begin{algorithm}[H]
  \fRecv{START()}{}{
    $parent_{i,i} \gets i;$
    $children_{i,i} \gets \emptyset;$
    $expected\_msg_{i,i} \gets \abs{neighbors_i};$
    \fNewLine
    \fSend{$j$}{$neighbors_i$}{GO($i, data$)}{$p_j$}
  }

  \fNewLine

  \fRecv{GO($root, data$)}{$p_j$}{
    \uIf{$parent_{root,i} = \perp$}{
      $parent_{root,i} \gets j;$
      $children_{root,i} \gets \emptyset;$
      $expected\_msg_{root,i} \gets \abs{neighbors_i \backslash \{j\}};$
      \fNewLine
      \uIf{$expected\_msg_{root,i}=0$}{
        \fSend{}{}{BACK($root, \{i,v_i\}$)}{$p_j$}
      } \fElse{\fSend{$k$}{$neighbors_i \backslash \{j\}$}{GO($root, data$)}{$p_k$}}
    } \fElse{\fSend{}{}{BACK($root, \emptyset$)}{$p_j$}}
  }

  \fNewLine

  \fRecv{BACK($root, val\_set$)}{$p_j$}{
    $expected\_msg_{root,i} \gets expected\_msg_{root,i}-1$\;
    $val\_set_{root,j} \gets val\_set$\;
    \fIf{$val\_set \ne \emptyset$}{$children_{root,i} \gets children_{root,i} \cup \{j\}$}
    \If{$expected\_msg_{root,i}=0$}{
      $val\_set \gets (\bigcup_{x \in children_i}val\_set_{root, x}) \cup \{(i, v_i)\}$\;
      \uIf{$i \ne root$}{
        \fSend{}{}{Back($root, val\_set$)}{$parent_{root, i}$}
      } \fElse{$p_i$ can compute $f_i(val\_set)$}
    }
  }

  \caption{Parellel PIF}
\end{algorithm}


\section{Traversal}

\begin{algorithm}[H]
  \fRecv{START()}{}{
    \fSend{}{}{GO($i, token$)}{$p_i$}
  }

  \fNewLine

  \fRecv{GO(pre, $token$)}{}{
    \If{$parent_i = \perp$}{
      $parent_i \gets pre$\;
      $unreached_i \gets neighbors_i \backslash \{pre\};$
    }
    \eIf{$unreached_i \ne \emptyset$}{
      \tcc{pop() will return an element in the set then remove it from the set.}
      $next \gets pop(unreached_i)$\;
      \fSend{}{}{GO($i, token$)}{$p_{next}$}
    }{
      \eIf{$i = parent_i$}{
        terminate()\;
      }{
        \fSend{}{}{GO($i, token$)}{$p_{parent_i}$}
      }
    }
  }

  \caption{Tarry's Traversal}
\end{algorithm}

\subsection{Lemma 1 Eventually every node is visited.}
1. All the processes are connected.\\
2. The root sends the token to all neighbors.\\
3. Other processes forward the token through all the edges.\\
4. Eventually every process receives the token.\\
5. Eventually every node is visited.\\
Q.E.D.


\subsection{Lemma 2 No two nodes are visited simultaneously.}
1. Only one node will receive START().\\
2. Nodes will only forward the token to one node at a time.\\
3. No two nodes are visited simultaneously.\\
Q.E.D.

\end{document}
