\documentclass{article}

\usepackage[a4paper, total={6.5in, 11in}]{geometry}
\setlength{\parindent}{0em}

\usepackage{latex/common}

\newcommand{\under}[2]{\mathop{#1}\limits_{#2}}
\newcommand{\s}{\sum\limits_{i=1}^{n}}
\newcommand{\p}[2]{\frac{\partial #1}{\partial #2}}
\newcommand{\pp}[3]{\frac{{\partial}^2 #1}{\partial #2 \partial #3}}
\newcommand{\B}[1]{\left\{\begin{aligned}#1\end{aligned}\right.}
\newcommand{\E}{\mathrm{E}}
\newcommand{\V}{\mathrm{V}}

\begin{document}

\section{P1}

\begin{center}
  $
  \begin{aligned}
    \under{minimize}{w} \;  & \frac{1}{2} ||Xw-y||^2_2 + \lambda 1^T q \\
    subject \; to \;        & q>=0 \\
                            & -q \le w \le q
  \end{aligned}
  $ \\
\end{center}

Assume that
\begin{center}
  $
  \begin{aligned}
    J = \frac{1}{2} ||Xw-y||^2_2 + \lambda 1^T q & \qquad (1) \\
    \Matrix{H}{h_1 \\ h_2 \\ h_3}                & \qquad (2) \\
    \Matrix{A}{0, \; -w, \; w}                   & \qquad (3) \\
  \end{aligned}
  $ \\
\end{center}

The KKT conditions are: \\
\begin{center}
$
  \B{
    \nabla (J+H^T(A-q)) = 0 \\
    H^T(A-q) = 0 \\
    H \ge 0 \\
  }
$ \\
\end{center}

According to \\
\begin{center}
  $
  \begin{aligned}
                         & \nabla (J+H^T(A-q))                                             & = 0          & \\
                         & \Matrix{}{X^T(Xw-y) - h_2 + h_3 \\ \lambda 1 - h_1 - h_2 - h_3} & = 0          & \\
    X^T(Xw-y) + \lambda1 & = h_1 + 2 h_2                                                   & \ge 0        & \qquad by (3)\\
    X^T(Xw-y) - \lambda1 & = -h_1 - 2 h_3                                                  & \le 0        & \qquad by (3)\\
                         & |X^T(Xw-y)|                                                     & \le \lambda1 & \\
  \end{aligned}
  $
\end{center}

If $q > 0$ and $w > 0$ then $h_1 = h_2 = 0$, so $X^T(Xw-y) = - \lambda1$ \\
If $q > 0$ and $w < 0$ then $h_1 = h_2 = h_3 = 0$, so $X^T(Xw-y) = \lambda1$ \\
If $q = 0$ then $w = 0$, and $h_1 > 0$, $h_2 > 0$, $h_3 > 0$, so $|X^T(Xw-y)| < \lambda1$ \\

\section{P2}

\begin{center}
  $
  \begin{aligned}
    f(x) = \frac{1}{2} a (x-\mu)^2+b, a>0 & \\
    ||\nabla f(x) - \nabla f(y)||_2 &= ||a(x-\mu) - a(y-\mu)||_2 \\
                                    &= ||a(x-y)||_2 \\
                                    &= a||(x-y)||_2 \\
                             \gamma &= a \\
    g(w) = \frac{1}{2} ||Xw-y||_2^2 & \\
    ||\nabla g(u) - \nabla g(v)||_2 &= ||X^T(Xu-y) - X^T(Xv-y)||_2 \\
                                    &= ||X^TX(u-v)||_2 \\
  \end{aligned}
  $
\end{center}

Due to $X^TX$ is a symmetric matrix, so $X^TX = Q \Lambda Q^{-1} = Q \Lambda Q^T$, where $\Lambda$ is a diagonal matrix whose entries are the eigenvalues of $X^TX$

\begin{center}
  $
  \begin{aligned}
    ||\nabla g(u) - \nabla g(v)||_2 &= ||X^TX(u-v)||_2 \\
                                    &= \sqrt{(X^TX(u-v))^T X^TX(u-v)} \\
                                    &= \sqrt{(u-v)^TX^TXX^TX(u-v)} \\
                                    &= \sqrt{(u-v)^T Q \Lambda Q^T Q \Lambda Q^T (u-v)} \\
                                    &= \sqrt{(u-v)^T Q \Lambda Q^{-1} Q \Lambda Q^T (u-v)} \\
                                    &= \sqrt{(u-v)^T Q \Lambda^2 Q^T (u-v)} \\
                                    &= \sqrt{\sum_i \lambda_i^2 (u-v)^T q_i q_i^T (u-v)} \\
                                    &\le max_i(\lambda_i) ||(u-v)||_2 \\
                             \gamma &= max_i(\lambda_i)
  \end{aligned}
  $
\end{center}



\end{document}
